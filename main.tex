%\documentclass{beamer}
% because the year is likely >2015, might want to support widescreen:
\documentclass[aspectratio=169]{beamer}
% (the default is 43, i.e. 4:3)


\usepackage[T1]{fontenc}
\usepackage[utf8]{inputenc}
\usepackage{lmodern}

\usepackage{tikz}\usetikzlibrary{positioning}
\usepackage{xcolor}

\usepackage{listings}

% Styling. Highly recommend using the metropolis theme, which has many small
% `features' that help a lot with readability and comprehensibility.

\usetheme{metropolis}

% Careful with the colors!
%\setbeamercolor{normal text}{fg=black!75,bg=black!2}
%\setbeamercolor{alerted text}{fg=green!80!black}
%\setbeamercolor{palette primary}{fg=red!10,bg=violet!50!black}
%\setbeamercolor{palette primary}{bg=red!40!black}
%\setbeamercolor{palette primary}{bg=green!50!yellow!80!black}

\usepackage{FiraSans}
% some other font packages:
%\usepackage[defaultsans]{droidsans}
%\usepackage[sfdefault]{cabin}
%\usepackage[math]{iwona}
%\usepackage[sfdefault,light]{roboto}

% colored underline, with Beamer overlay support                                                                                                                                                                                                                                                                                                                                                                                         
% usage: \cul{x} or \cul[blue]{x} or \cul<2->{x} or \cul<2->[blue]{x}                                                                                                                                                                                                                                                                                                                                                                    
\newcommand<>{\cul}[2][red]{%
  % change underline dimentions: https://tex.stackexchange.com/a/167957/25264
  \fontdimen8\textfont3=0.75pt%
  % colored underline: https://tex.stackexchange.com/a/9477/25264
  % transparent underline: https://tex.stackexchange.com/a/45601/25264
  % switch between colored and transparent: http://mirrors.ibiblio.org/CTAN/macros/latex/contrib/beamer/doc/beameruserguide.pdf sections 9.3 and 9.6.1
  \alt#3%
      {\color{#1}\underline{{\color{black}#2}}\color{black}}%
      {\transparent{0.0}\underline{{\transparent{1.0}#2}}\transparent{1.0}}%
}

\author{Jiří Klepl}
\title{Type Inference and Polymorphism for C} %really change this!
\date{September 2020} %change this too
% better alternative:
%\date{MFF UK, September 2020}
% also possible:
%\date{}

\begin{document}
\maketitle

% Start with roughly 3 slides that work as an `abstract'. Give a totally brief
% overview of what is the motivation for the thesis, how you approached it, and
% what is the main result.
%
% The point of this is that your problem, approach and results, if coherent,
% will deliver a message to the commitee that your thesis MAKES SENSE. The
% sooner they know, the better.
%
% I follow with a demonstration that shows how I would present and defend why I
% developed this `template' slideshow.

\begin{frame}{The C Language}
\begin{itemize}
\item[\textcolor{green}{\textbullet}] Valued for its simplicity and code transparency
\item[\textcolor{green}{\textbullet}] One of the most influential languages in systems
development
\item[\textcolor{green}{\textbullet}] The language of kernel development
\end{itemize}

\begin{itemize}
\item[\textcolor{red}{\textbullet}] No support for polymorphism
\item[\textcolor{red}{\textbullet}] Generic constructions must be simulated (macros, void*)
\end{itemize}
\end{frame}


\begin{frame}[fragile, t]{Generic Programming in C (example from BSD: queue.h)}

\begin{lstlisting}[language=C,showstringspaces=false,basicstyle=\tt\small,commentstyle=\color{green!50!black},keywordstyle=\bfseries\color{blue!50!black},stringstyle=\color{red!50!black},escapeinside={\%*}{*)}]
SLIST_HEAD(%*\alert<2>{slisthead}*), entry) head = SLIST_HEAD_INITIALIZER(head);

struct entry {
 ...
 SLIST_ENTRY(entry) entries; /* Singly-linked List. */
 ...
} *n1;

n1 = malloc(%*\alert<2>{sizeof(struct entry)}*));
SLIST_INSERT_HEAD(&head, n1, %*\alert<2>{entries}*)); /* Insert at the head. */
\end{lstlisting}

\uncover<2>{\alert{We have to provide type-relevant information (this could be inferred).}}
\end{frame}

\begin{frame}[fragile]{Example from BSD: queue.h}
\begin{lstlisting}[language=C,showstringspaces=false,basicstyle=\tt\small,commentstyle=\color{green!50!black},keywordstyle=\bfseries\color{blue!50!black},stringstyle=\color{red!50!black}]
SLIST_HEAD(int_slist, int_entry) head = SLIST_HEAD_INITIALIZER(head);

struct int_entry {
    SLIST_ENTRY(int_entry) entries;
    int value;
} n1;

n1 = malloc(sizeof(struct int_entry));
n1->value = 1;
SLIST_INSERT_HEAD(&head, n1, entries); 
\end{lstlisting}
\end{frame}

\begin{frame}[fragile]{Our approach}
\begin{itemize}
    \item Minimal language extension
    \item Hindley-Milner-style polymorphism and inference
    \item Type classes for overloading
\end{itemize}

\pause

\begin{lstlisting}[language=C,showstringspaces=false,basicstyle=\tt\small,commentstyle=\color{green!50!black},keywordstyle=\bfseries\color{blue!50!black},stringstyle=\color{red!50!black}]
a head = empty(); // the type is inferred
insert_head(&head, (int)1);
\end{lstlisting}
\end{frame}

\begin{frame}[fragile, t]{Polymorphic list implementation}
\begin{lstlisting}[language=C,showstringspaces=false,basicstyle=\tt\small,commentstyle=\color{green!50!black},keywordstyle=\bfseries\color{blue!50!black},stringstyle=\color{red!50!black}]
<a : b ~ struct list : <a> > // type variables and derived types
struct list {
    b *next;
    a value;
};

<a : b ~ struct list : <a> >
void insert_head(b **head, a value) {
    b *node = new(); // required type for new() is inferred
    node->next = *head;
    node->value = value;
    *head = node;
}
\end{lstlisting}
\end{frame}

\begin{frame}[fragile]{Simple helper functions}
\begin{lstlisting}[language=C,showstringspaces=false,basicstyle=\tt\small,commentstyle=\color{green!50!black},keywordstyle=\bfseries\color{blue!50!black},stringstyle=\color{red!50!black}]
<a, b : c ~ b : <a> >
c *empty() { 
    return (c *)NULL; 
}

<a>
a *new() {
    return (a *)malloc(sizeof(a));
}
\end{lstlisting}
\end{frame}

\begin{frame}{Results}
\begin{itemize}
\item Proof-of-concept compiler to C
\item Functionality partially overlaps with C++ templates
\item Practical programs that work with CHM
\item Identified future challenges:
\begin{itemize}
    \item implicit conversions (subtyping, constantness, ...)
    \item type of \texttt{struct} accessors
\end{itemize}
\end{itemize}
\end{frame}

\begin{frame}[fragile]{Example output}
Compiler output: normal C code, can be compiled with \texttt{gcc}
\begin{lstlisting}[language=C,showstringspaces=false,basicstyle=\tt\tiny,commentstyle=\color{green!50!black},keywordstyle=\bfseries\color{blue!50!black},stringstyle=\color{red!50!black}]
typedef struct TA7TC4list3int TA7TC4list3int;
struct TA7TC4list3int {  TA7TC4list3int * next; int value;  };

TA7TC4list3int * __new_TF018TP14TA7TC4list3int()
{  return (TA7TC4list3int *) malloc(sizeof(TA7TC4list3int));  }

void __insert_head_TF25TP18TP14TA7TC4list3intint7TC4Void(TA7TC4list3int * * head,
                                                         int value)
{   TA7TC4list3int * node = __new_TF018TP14TA7TC4list3int();
    node->next = *head;
    node->value = value;
    *head = node;  }

TA7TC4list3int * __nullptr_TF018TP14TA7TC4list3int()
{ return (TA7TC4list3int *) (void *) 0; }

TP14TA7TC4list3int head = __nullptr_TF018TP14TA7TC4list3int();
__insert_head_TF25TP18TP14TA7TC4list3intint7TC4Void(&head, (int) 1);
\end{lstlisting}
\end{frame}

\section{How does the CHM compiler work?}

\begin{frame}{Implementation overview}
\begin{itemize}
    \item Parsing (uses modified Language.C)
    \item Conversion to $\lambda_C$
    \item Type Inference (uses modified THIH)
    \item Monomorphization
    \begin{itemize}
        \item Instantiation
        \item Mangling (`TP', `TF', structs, primitive types)
    \end{itemize}
    \item Code output
    \item Compiling (uses gcc)
\end{itemize}
\end{frame}

\begin{frame}[fragile]{Representing C types using $\lambda_C$ conversion}
\begin{itemize}
\item Primitive types and structs (unions) modeled directly:
\begin{itemize}
    \item $[[t]]_{\lambda_c} = \alert<2>{t}$
\end{itemize}
\item Pointers and functions:
\begin{itemize}
    \item $[[t\ *]]_{\lambda_c} = \alert<2>{(*)}\ [[t]]_{\lambda_c}$
    \item $[[t(p_1, \dots p_n)]]_{\lambda_c} = \alert<2>{(}[[p_1]]_{\lambda_c}\alert<2>{,} \dots\alert<2>{,} [[p_n]]_{\lambda_c}\alert<2>{)} \alert<2>{\to} [[t]]_{\lambda_c}$
\end{itemize}
\end{itemize}
\end{frame}

\begin{frame}[fragile]{Representing type dependencies inside C constructs using $\lambda_C$ conversion}
\begin{itemize}
\item An example for C addition (functions and initializations more complex)
\begin{itemize}
    \item $[[a + b]]_{\lambda_c} = (+)\ [[a]]_{\lambda_c}\ [[b]]_{\lambda_c}$
    \item $(+) : \forall \alpha . Num(\alpha) \Rightarrow \alpha \to \alpha \to \alpha$
\end{itemize}
\item Challenges
\begin{itemize}
    \item No implicit casts (we can use explicit casts)
    \item Const semantics with the C syntax do not fit type inference rules
    \item C dot operator with a given field syntactically equivalent to a function
    \begin{itemize}
        \item To preserve syntax we have to make limitations on field types
    \end{itemize}
\end{itemize}
\end{itemize}
\end{frame}

\begin{frame}[fragile]{Extending the language (polymorphic functions and structures)}
\begin{lstlisting}[language=C,showstringspaces=false,basicstyle=\tt\small,commentstyle=\color{green!50!black},keywordstyle=\bfseries\color{blue!50!black},stringstyle=\color{red!50!black}]
<a>
a copy(a value) { return value; }

<a>
struct container { a value; };

<a : b ~ container : <a> >
b wrap(a value) {
    b wrapper;
    wrapper.value = value;
    return wrapper;
}
\end{lstlisting}
\end{frame}

\begin{frame}[fragile]{User-defined type classes and instances}
\begin{columns}[t]

\begin{column}{0.5\textwidth}
\begin{lstlisting}[language=C,showstringspaces=false,basicstyle=\tt\small,commentstyle=\color{green!50!black},keywordstyle=\bfseries\color{blue!50!black},stringstyle=\color{red!50!black}]
<a : Div<a> >
class SafeDiv<a> { 
a safe_div(a left, 
           a right);
}

instance SafeDiv<float> {
float safe_div(float left,
               float right) {
    return left / right;
}
}
\end{lstlisting}
\end{column}

\hspace{-40pt}
\vrule
\hspace{5pt}

\begin{column}{0.5\textwidth}
\begin{lstlisting}[language=C,showstringspaces=false,basicstyle=\tt\small,commentstyle=\color{green!50!black},keywordstyle=\bfseries\color{blue!50!black},stringstyle=\color{red!50!black}]
instance SafeDiv<int> { 
int safe_div(int left,
             int right) { 
    if (right != 0)
        return left / right; 
    else return 0;
}
}
\end{lstlisting}
\end{column}

\end{columns}
\end{frame}

% Ending frame is important -- in the rare case that you actually manage to
% finish the presentation on time, it avoids the embarassing seconds of silence
% between the time when you finish and the moment when committee detects that
% something is wrong and they should continue.
\begin{frame}[plain]
\centering
{\Large\bfseries Thank you for attention!}
\end{frame}
\end{document}
